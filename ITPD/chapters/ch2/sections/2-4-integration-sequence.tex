% !TEX root = ../../../main.tex

\section{Sequence of Component/Function Integration}
In this section we are going to describe the order of integration of PowerEnJoy's components. As already mentioned, the adopted integration strategy is mainly bottom-up, so, as notation, an arrow going from component C1 to component C2 means that C1 must have been already developed and unit tested in order to be integrated with C2.

\subsection{Test Plan 1}
Referring to the critical-module-first logic adopted in the integration strategy, CarController is the most important component in the system, so we start to integrate it with the other related components.
\begin{figure}[H]
	\centering
	\includegraphics[width=0.5\textwidth]{TestPlan/TP1}
	\caption[Test Plan 1 diagram]{The picture above shows the Test Plan 1 diagram. In this Test Plan, MapController is integrated with CarController.}
	\label{fig:TestPlan-1}
\end{figure}

\subsection{Test Plan 2}
For the same reasons explained above in Test Plan 1, we can integrate RecoveryHelper with CarController using the threads strategy explained above.
\begin{figure}[H]
	\centering
	\includegraphics[width=0.5\textwidth]{TestPlan/TP2}
	\caption[Test Plan 2 diagram]{The picture above shows the Test Plan 2 diagram. In this Test Plan, RecoveryHelper is integrated with CarController.}
	\label{fig:TestPlan-2}
\end{figure}

\subsection{Test Plan 3}
The core of the system is surely the relation between CarController, BookingController and RentalController, so the next step is to integrate CarController with the other two components.
\begin{figure}[H]
	\centering
	\includegraphics[width=0.5\textwidth]{TestPlan/TP3}
	\caption[Test Plan 3 diagram]{The picture above shows the Test Plan 3 diagram. In this Test Plan, CarController is integrated with BookingController and RentalController.}
	\label{fig:TestPlan-3}
\end{figure}

\subsection{Test Plan 4}
We can now integrate RentalController with BookingController to have core functionalities of the system fully tested and integrated.
\begin{figure}[H]
	\centering
	\includegraphics[width=0.2\textwidth]{TestPlan/TP4}
	\caption[Test Plan 4 diagram]{The picture above shows the Test Plan 4 diagram. In this Test Plan, RentalController is integrated with BookingController.}
	\label{fig:TestPlan-4}
\end{figure}

\subsection{Test Plan 5}
By threads we can integrate PaymentHelper with BoookingController and RentalController, following the bottom-up strategy in order to test the full system.
\begin{figure}[H]
	\centering
	\includegraphics[width=0.5\textwidth]{TestPlan/TP5}
	\caption[Test Plan 5 diagram]{The picture above shows the Test Plan 5 diagram. In this Test Plan, PaymentHelper is integrated with BookingController and RentalController.}
	\label{fig:TestPlan-5}
\end{figure}

\subsection{Test Plan 6}
For the same reasons explained above in Test Plan 5, we can also integrate by threads NotificationHelper with CarController and BookingController.
\begin{figure}[H]
	\centering
	\includegraphics[width=0.5\textwidth]{TestPlan/TP6}
	\caption[Test Plan 6 diagram]{The picture above shows the Test Plan 6 diagram. In this Test Plan, NotificationHelper is integrated with CarController and BookingController.}
	\label{fig:TestPlan-6}
\end{figure}

\subsection{Test Plan 7}
Finally we can test the integration between all core components and the Router component, which is needed to start every process/request in the system.
\begin{figure}[H]
	\centering
	\includegraphics[width=0.5\textwidth]{TestPlan/TP7}
	\caption[Test Plan 7 diagram]{The picture above shows the Test Plan 7 diagram. In this Test Plan, all core components are integrated with the Router component.}
	\label{fig:TestPlan-7}
\end{figure}

\subsection{General Overview}
\begin{figure}[H]
	\centering
	\includegraphics[width=1\textwidth]{TestPlan/all}
	\caption[General Overview diagram]{The picture above shows the general overview of how all components are integrated in PowerEnJoy's system.}
	\label{fig:General Overview}
\end{figure}