\section{Use case model}

\begin{figure}[H]
	\centering
	\includegraphics[width=\textwidth]{use-case-model}
	\caption[Use case model]{The diagram above represents the use case model of the PowerEnJoy project. The actors included are the unregistered user, the user and the PowerEnJoy system.}
	\label{fig:use-case-model}
\end{figure}

\subsection{Use case description 1 - Sign up}
\begin{labeling}{use-case-desc-1}
		\item[\textbf{Name}] Sign up
		\item[\textbf{Actors}] User
		\item[\textbf{Entry conditions}] There are no entry conditions.
		\item[\textbf{Flow of events}]
			\begin{itemize}
				\item[]
				\item The user accesses the homepage of PowerEnJoy or opens the app.
				\item The user inserts his/her credentials in the registration form.
				\item The user clicks on the sign up button.
				\item The system processes the registration of the user.
			\end{itemize}
		\item[\textbf{Exit conditions}] The user is successfully registered to the system.
		\item[\textbf{Exceptions}]
			\begin{itemize}
				\item[]
				\item Credentials provided by the user are not correct. In this case the system notifies the user of the error and let him/her to input again his/her credentials. 
				\item User is already registered. In this case the system notifies the user of the impossibility to register.
			\end{itemize}
	\end{labeling}

\subsection{Use case description 3 - Sign in}
\begin{labeling}{use-case-desc-2}
		\item[\textbf{Name}] Sign in
		\item[\textbf{Actors}] User
		\item[\textbf{Entry conditions}] The user has to be already registered.
		\item[\textbf{Flow of events}]
			\begin{itemize}
				\item[]
				\item The user accesses the homepage of PowerEnJoy or opens the app
				\item The user inputs his email and password
				\item The user clicks on the sign in button
				\item The system redirects the user on his personal page
			\end{itemize}
		\item[\textbf{Exit conditions}] The user is successfully redirected to his personal page.
		\item[\textbf{Exceptions}]
			\begin{itemize}
				\item[]
<<<<<<< HEAD
				\item Email or password provided by the user are not correct. In this case the system notifies the user of the error and let him/her to input again his/her credentials. 
=======
				\item Credentials provided by the user are not correct. In this case the system notifies the us.
				\item The user doesn’t receive the PIN via mail. In this case the user can ask to the system to send again the mail with the PIN. 
>>>>>>> 996b3693a1c762d2f95af4817b46a466fb0b512f
			\end{itemize}
	\end{labeling}
	
\subsection{Use case description 3}